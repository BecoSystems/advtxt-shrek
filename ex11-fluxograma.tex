%%%%%%%%%%%%%%%%%%%%%%%%%%%%%%%%%%%%%%%%%%%%%%%%%%%%%%%%%%%%%%%%%%%%%%%%%%%%%%%%%%%%%%%%
% Criação de Fluxograma usando LaTeX
%
% Assunto: escrever aqui um comentário com uma
%          breve explicação do exercício
%
% Autores:
%     Nome do Aluno 1
%     Nome do Aluno 2
%     Nome do Aluno 3
%
% Coordenação:
%     Prof. Dr. Ruben Carlo Benante
%
% Data: 2024-04-25
%%%%%%%%%%%%%%%%%%%%%%%%%%%%%%%%%%%%%%%%%%%%%%%%%%%%%%%%%%%%%%%%%%%%%%%%%%%%%%%%%%%%%%%%


%%%%%%%%%%%%%%%%%%%%%%%%%%%%%%%%%%%%%%%%%%%%%%%%%%%%%%%%%%%%%%%%%%%%%%%%%%%%%%%%%%%%%%%%
% Para gerar o PDF use o comando make com o makefile configurado:
%
%    $ make ext-programa2-benante-sobrenome1-sobrenome2.pdf
%
% O conteúdo do makefile é composto dos 3 seguintes comandos que ficam assim automatizados:
%    $ pdflatex exN-fluxograma.tex -o exN-fluxograma.pdf
%    $ bibtex biblio
%    $ pdflatex exN-fluxograma.tex -o exN-fluxograma.pdf


%%%%%%%%%%%%%%%%%%%%%%%%%%%%%%%%%%%%%%%%%%%%%%%%%%%%%%%%%%%%%%%%%%%%%%%%%%%%%%%%%%%%%%%%
% preambulo %%%%%%%%%%%%%%%%%%%%%%%%%%%%%%%%%%%%%%%%%%%%%%%%%%%%%%%%%%%%%%%%%%%%%%%%%%%%
\documentclass[a4paper,12pt]{article} %twocolumn
\usepackage[left=2.5cm,right=2cm,top=2.5cm,bottom=2cm]{geometry}
\usepackage[utf8]{inputenc} % letras acentuadas
\usepackage[portuguese]{babel} % tradução de títulos
\usepackage[colorlinks]{hyperref}
\usepackage{tikz} % para adicionar fluxogramas
\usepackage{algorithm} % ambiente para índice de algoritmos
\usepackage{algpseudocode} % fonte e estilo do algoritmo
\usepackage{graphicx} % permite adicionar imagens
\usepackage{indentfirst} % indenta o primeiro parágrafo também
\usepackage{url} % permite adicionar links de URLs e emails
% \usepackage{natbib}
%[noend]

\DeclareUrlCommand\email{\urlstyle{mm}} % comando para email bonito
\floatname{algorithm}{Algoritmo} % tradução da palavra algoritimo no ambiente de índice

\usetikzlibrary{shapes.geometric, shapes.symbols,arrows} % ajuste do tikz para incluir formas e setas

%%%%%%%%%%%%%%%%%%%%%%%%%%%%%%%%%%%%%%%%%%%%%%%%%%%%%%%%%%%%%%%%%%%%%%%%%%%%%%%%%%%%%%%%
% capa %%%%%%%%%%%%%%%%%%%%%%%%%%%%%%%%%%%%%%%%%%%%%%%%%%%%%%%%%%%%%%%%%%%%%%%%%%%%%%%%%
\title{Fluxograma: Shrek The Adventure}
\author{Gilmar Lopes \\ Vinicius Medeiros \\ Luis Guilherme}

\begin{document}

\maketitle

%%%%%%%%%%%%%%%%%%%%%%%%%%%%%%%%%%%%%%%%%%%%%%%%%%%%%%%%%%%%%%%%%%%%%%%%%%%%%%%%%%%%%%%%
% definicao dos blocos do fluxograma (tikz) %%%%%%%%%%%%%%%%%%%%%%%%%%%%%%%%%%%%%%%%%%%%

\tikzstyle{line} = [draw, -latex']
\tikzstyle{startend} = [draw, ellipse,fill=red!20, minimum height=2em, node distance=1.55cm]
\tikzstyle{print} = [tape, fill=blue!20, draw, draw=black, minimum width=3cm, minimum height=1.4cm, text width=4.5em, text centered, tape bend top=none, tape bend height=0.2cm, node distance=1.55cm]
\tikzstyle{input} = [trapezium, trapezium left angle=60, trapezium right angle=90, minimum width=3cm, minimum height=1cm, text centered, draw=black, fill=blue!30, node distance=1.95cm]
\tikzstyle{process} = [rectangle, minimum width=3cm, minimum height=1cm, text centered, draw=black, fill=orange!30, node distance=1.55cm]

\tikzstyle{block} = [rectangle, draw, fill=blue!20, text width=5em, text centered, rounded corners, minimum height=4em, node distance=1.55cm]
\tikzstyle{decisionb} = [diamond, draw, fill=blue!20, text width=4.5em, text badly centered, inner sep=0pt, node distance=1.55cm]
\tikzstyle{decision} = [diamond, minimum width=3cm, minimum height=1cm, text centered, draw=black, fill=green!30, node distance=2.25cm]
\tikzstyle{empty} = [circle, fill=white, minimum width=0.01mm, node distance=2.55cm]

%%%%%%%%%%%%%%%%%%%%%%%%%%%%%%%%%%%%%%%%%%%%%%%%%%%%%%%%%%%%%%%%%%%%%%%%%%%%%%%%%%%%%%%%
% resumo %%%%%%%%%%%%%%%%%%%%%%%%%%%%%%%%%%%%%%%%%%%%%%%%%%%%%%%%%%%%%%%%%%%%%%%%%%%%%%%

\begin{abstract}

\textbf{Assunto:} Programa ex11

% descrever em poucas palavras seu projeto aqui

O projeto é um jogo adventure text com duas escolhas (a primeira, um objeto, a segunda, um verbo. Se não escolhidos corretamente, o jogador morre). Trabalho em duplas ou triplas. Neste artigo iremos apresentar o seu fluxograma completo
% e (opcionalmente) o seu algoritmo.

Após a modelagem do fluxograma e desenvolvimento da lógica de programação em algoritmo,
o programa será implementado na Linguagem de Programação \texttt{C}


\textbf{Local:} Escola Politécnica de Pernambuco - UPE/POLI

\textbf{Órgão Financiador:} N/A

\textbf{Caracterização:} Modelagem, Projeto e Implementação de Software em Linguagem \texttt{C}

% Este é o fim do resumo.

\end{abstract}

%%%%%%%%%%%%%%%%%%%%%%%%%%%%%%%%%%%%%%%%%%%%%%%%%%%%%%%%%%%%%%%%%%%%%%%%%%%%%%%%%%%%%%%%
% artigo %%%%%%%%%%%%%%%%%%%%%%%%%%%%%%%%%%%%%%%%%%%%%%%%%%%%%%%%%%%%%%%%%%%%%%%%%%%%%%%
% seção de introdução %%%%%%%%%%%%%%%%%%%%%%%%%%%%%%%%%%%%%%%%%%%%%%%%%%%%%%%%%%%%%%%%%%
\section{Introdução}

% Descrever melhor seu projeto aqui
O programa é dividido em 4 funções, sendo elas a \textbf{main}, a \textbf{void menu}, \textbf{void jogo} e \textbf{void jogo 2}. A função principal possui uma estrutura de escolha, sendo ela (1), chama a função menu,(2) encerrar o programa, e uma medida de segurança que rejeita outras possibilidades de digitação do usuário através de um looping, a função main em si possui também um looping através do "do while" que nos permite \textbf{retornar a função através do menu}. O menu também é montado em forma de loop com "do while" ele no entanto tem três opções,1, 2 e 3.

A opção chama a função jogo, a 2 fecha o menu e retorna para a main, a 3 mostra informações relativas aos desenvolvedores e depois retorna a main. Há também uma mesma estrutura de segurança através do while que impede erro por digitação de chat errado, inclusive, em todas as funções há uma variável carácter que vá ser usada para aplicar o "switch". Ao selecionar a 2 o menu é \textbf{encerrado} e voltamos para main, se 3, os nomes dos desenvolvedores aparece e voltamos para o menu, se 1 iniciarmos a função jogo. Função jogo é composta da mesma estrutura de seleção que as outras, \textbf{uma variável seletora}, um texto que entra em contato com o usuário e lhe oferece as opções, nesse caso 2, referentes a escolha de um objeto, se for a opção correta o usuário vai para a parte 2 ou seja, a função jogo 2 é chamada, se errado imprime uma frase de derrota e voltamos para o menu. Jogo 2, novamente 2 escolhas, dessa vez referente a ação, se correta o usuário recebe o bom final e a frase de vitória e voltamos para o menu, se errado, um final ruim, frase de derrota e voltamos pra o menu.

O programa será modelado em \textit{fluxograma} em uma primeira fase, em seguida
sua lógica será desenvolvida em formato de \textit{algoritmo}, para então
na terceira fase ser implementado em Linguagem de Programação \texttt{C}.

%%%%%%%%%%%%%%%%%%%%%%%%%%%%%%%%%%%%%%%%%%%%%%%%%%%%%%%%%%%%%%%%%%%%%%%%%%%%%%%%%%%%%%%%
% seção de objetivos %%%%%%%%%%%%%%%%%%%%%%%%%%%%%%%%%%%%%%%%%%%%%%%%%%%%%%%%%%%%%%%%%%%
\section{Fluxograma}

% adicionar aqui o fluxograma

\begin{tikzpicture}
    % primeira parte do fluxogrma
    \node (inicio) [startend] {Início};
    \node (txta) [print, below of=inicio, node distance=2cm] {Shrek The Adventure};
    \node (dec1) [decision, below of=txta, node distance=3cm] {decisão $=$ 1};
    \node (encerrando1) [print, right of=dec1, node distance=4.cm] {encerrando...};
    \node (menu) [process, below of=dec1, node distance=3cm] {Menu de Opções};
    \node (dec2) [decision, below of=menu, node distance=3.5cm] {decisão $=$ 1, 2 ou 3};
    \node (txtenc) [print, left of=dec2, node distance=4.5cm] {Encerrando...};
    \node (txtinfo) [print, right of=dec2, node distance=4.5cm] { Devs info};
    \node (txt1) [print, below of=dec2, node distance=3.5cm] {texto};
    \node (obj) [input, below of=txt1, node distance=2.5cm] {Lanterna};
    \node (dec3) [decision, below of=obj, node distance=3cm] {decisão $=$ 1};
    \node (txtmorte) [print, right of=dec3, node distance=4cm] {Morreu}
    \node (fimfinal) [startend, below of=txtmorte, node distance=2cm] {Fim};
    \node (1inicio) [startend, below of=dec3, node distance=3cm] {pt2};

% \node (vazio1) [empty, right of=fim, node distance=4cm] {};
% Desenhar as setas
    \path [line] (inicio) -- (txta);
    \path [line] (txta) -- (dec1);
    \path [line] (dec1) -- node[anchor=east] {Sim} (menu);
    \path [line] (dec1.east) -- node[anchor=east] {Não} (encerrando1);
    \path [line] (menu) -- (dec2);
    \path [line] (dec2) -- node[anchor=east] {2} (txtenc);
    \path [line] (dec2) -- node[anchor=west] {3} (txtinfo);
    \path [line] (dec2) -- node[anchor=east] {1} (txt1);
    \path [line] (txt1) -- (obj);
    \path [line] (obj) -- (dec3);
    \path [line] (dec3.east) -- node[anchor=north] {Não} (txtmorte.west);
    \path [line] (dec3) -- node[anchor=east] {Sim} (1inicio);
    \path [line] (txtmorte) -- (fimfinal);

\end{tikzpicture}

\clearpage % inicia próxima seção em nova página
%%%%%%%%%%%%%%%%%%%%%%%%%%%%%%%%%%%%%%%%%%%%%%%%%%%%%%%%%%%%%%%%%%%%%%%%%%%%%%%%%%%%%%%%
% seção de justificativa %%%%%%%%%%%%%%%%%%%%%%%%%%%%%%%%%%%%%%%%%%%%%%%%%%%%%%%%%%%%%%%
% \section{Algoritmo}

% adicionar aqui o algoritmo (opcional)


% \clearpage % inicia próxima seção em nova página
%%%%%%%%%%%%%%%%%%%%%%%%%%%%%%%%%%%%%%%%%%%%%%%%%%%%%%%%%%%%%%%%%%%%%%%%%%%%%%%%%%%%%%%%
% Autores %%%%%%%%%%%%%%%%%%%%%%%%%%%%%%%%%%%%%%%%%%%%%%%%%%%%%%%%%%%%%%%%%%%%%%%%%%%%%%
\section*{Detalhamento dos Autores}

%%%%%%%%%%%%%%%%%%%%%%%%%%%%%%%%%%%%%%%%%%%%%%%%%%%%%%%%%%%%%%%%%%%%%%%%%%%%%%%%%%%%%%%%
% Discentes %%%%%%%%%%%%%%%%%%%%%%%%%%%%%%%%%%%%%%%%%%%%%%%%%%%%%%%%%%%%%%%%%%%%%%%%%%%%
\subsection*{Discentes}

\begin{enumerate}
    \item \textbf{Nome Completo:} Fulano de Tal Um
    \begin{description}
        \item [Email:] \email{blabla@poli.br}
        \item [Endereço:]
        \item [Matrícula:]
        \item [CPF:]
        \item [RG:]
        \item [Telefone:]
        \item [Currículo Lattes:] \url{http://lattes.cnpq.br/nnnnn}
    \end{description}

    \item \textbf{Nome Completo:} Fulano de Qual Dois
    \begin{description}
        \item [Email:] \email{blabla@poli.br}
        \item [Endereço:]
        \item [Matrícula:]
        \item [CPF:]
        \item [RG:]
        \item [Telefone:]
        \item [Currículo Lattes:] \url{http://lattes.cnpq.br/nnnnn}
    \end{description}

    \item \textbf{Nome Completo:} Fulano de Como Três
    \begin{description}
        \item [Email:] \email{blabla@poli.br}
        \item [Endereço:]
        \item [Matrícula:]
        \item [CPF:]
        \item [RG:]
        \item [Telefone:]
        \item [Currículo Lattes:] \url{http://lattes.cnpq.br/nnnnn}
    \end{description}

    % \item \textbf{Nome Completo:} Fulano de tal
    % \begin{description}
        % \item [Email:] \email{blabla@poli.br}
        % \item [Endereço:]
        % \item [Matrícula:]
        % \item [CPF:]
        % \item [RG:]
        % \item [Telefone:]
        % \item [Currículo Lattes:] \url{http://lattes.cnpq.br/nnnnn}
    % \end{description}

%     \item \textbf{Nome Completo:} Fulano de tal
%     \begin{description}
%         \item [Email:] \email{blabla@poli.br}
%         \item [Endereço:]
%         \item [Matrícula:]
%         \item [CPF:]
%         \item [RG:]
%         \item [Telefone:]
%         \item [Currículo Lattes:] \url{http://lattes.cnpq.br/nnnnn}
%     \end{description}
\end{enumerate}


%%%%%%%%%%%%%%%%%%%%%%%%%%%%%%%%%%%%%%%%%%%%%%%%%%%%%%%%%%%%%%%%%%%%%%%%%%%%%%%%%%%%%%%%
% Docentes %%%%%%%%%%%%%%%%%%%%%%%%%%%%%%%%%%%%%%%%%%%%%%%%%%%%%%%%%%%%%%%%%%%%%%%%%%%%%
\subsection*{Docentes}

\begin{enumerate}
    \item \textbf{Nome Completo:} Ruben Carlo Benante
    \begin{description}
        \item [Email:] \email{rcb@upe.br}
        \item [Matrícula:] 11238-0
        \item [Currículo Lattes:] \url{http://lattes.cnpq.br/3366717378277623}
    \end{description}
\end{enumerate}


%%%%%%%%%%%%%%%%%%%%%%%%%%%%%%%%%%%%%%%%%%%%%%%%%%%%%%%%%%%%%%%%%%%%%%%%%%%%%%%%%%%%%%%%
% referências bibliográficas %%%%%%%%%%%%%%%%%%%%%%%%%%%%%%%%%%%%%%%%%%%%%%%%%%%%%%%%%%%
%\section*{Referências Bibliográficas}

% cite todos, mesmo os não referenciados %%%%%%%%%%%%%%%%%%%%%%%%%%%%%%%%%%%%%%%%%%%%%%%
\nocite{*}


%%%%%%%%%%%%%%%%%%%%%%%%%%%%%%%%%%%%%%%%%%%%%%%%%%%%%%%%%%%%%%%%%%%%%%%%%%%%%%%%%%%%%%%%
% se necessario %%%%%%%%%%%%%%%%%%%%%%%%%%%%%%%%%%%%%%%%%%%%%%%%%%%%%%
% troca autor and autor por autor & autor, na bibliografia. O dcu usa "and"
%\renewcommand{\harvardand}{\&} % troca and pro &. O dcu usa "and"

% Estilos de bibliografia %%%%%%%%%%%%%%%%%%%%%%%%%%%%%%%%%%%%%%%%%%%%%%%%%%%%%%%%%%%%%%
% \bibliographystyle{abnt-alf} % Estilo alfabético da ABNT. Opção [num] para estilo numérico
% \bibliographystyle{apalike}
% \bibliographystyle{dcu} %citacao como (autor and autor, ano). Parece apalike. Rev. Control. Automacao. Use com harvard
% \bibliographystyle{agsm} % padrao harvard fica (autor & autor ano).
\bibliographystyle{acm}

%%%%%%%%%%%%%%%%%%%%%%%%%%%%%%%%%%%%%%%%%%%%%%%%%%%%%%%%%%%%%%%%%%%%%%%%%%%%%%%%%%%%%%%%
% arquivo de banco de dados das referências %%%%%%%%%%%%%%%%%%%%%%%%%%%%%%%%%%%%%%%%%%%%
% renomear para o número do exercício correto
% o arquivo de bibliografia pode se chamar qualquer coisa, isso não muda o comando de gerar o PDF.
% Por exemplo para 'mybiblio.bib', use \bibliography{mybiblio} e os comandos pdflatex e bibtex continuam os mesmos identicos com exN.
\bibliography{biblio}

\end{document}
